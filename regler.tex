\documentclass[a4paper,12pt]{article}
\usepackage[T1]{fontenc}
\usepackage{lmodern}
\usepackage[utf8]{inputenc}

\title{Skitgubbe}
\date{28 augusti, 2013}
\author{Henrik Holmberg}

\begin{document}

\maketitle

- Antal spelare: 3-12

- Antal kortlekar: 2

\section{Spelets upplägg}
Spelet sker i två rundor.
Första rundan gäller det att ta stick som man ska spela med under andra rundan.
Andra rundan går ut på att bli av med de kort man har.

\section{Förberedelser}
En spelare blandar kortlekarna, delar ut tre kort till varje spelare och lägger resten i en hög i mitten.

\section{Första rundan}
Spelaren till vänster börjar rundan. Sedan lägger spelare i tur och ordning var sitt kort. Den som tar sticket får börja lägga kort till nästa stick.

\subsection{En spelares tur}
Spelaren lägger ut ett kort framför sig enligt följande regler och ska ha tre kort på hand så länge högen räcker (högre prio ju högre i listan):
\begin{itemize}
  \item Om spelaren har ett kort med samma valör som ett som redan ligger på bordet måste spelaren lägga detta (vid flera olika får spelare välja fritt bland dessa)
  \item Om spelaren har en eller flera jokrar på handen får spelaren välja att lägga valfritt antal tillsammans med ett annat kort. Jokrarna blir samma valör som det andra kortet spelare lägger (gäller dock inte som samma färg)
  \item Spelaren lägger ett kort från handen eller chansar och drar ett kort från högen (om det är en joker dras ett kort till)
\end{itemize}

\subsection{Vem som tar sticket}
När alla har lagt bestäms vem som får sticket med följande regler (högre prio ju högre i listan):
\begin{itemize}
  \item Om det finns två kort av samma valör och färg får de spelarna inklusive andra som lagt samma valör spela
  \item Om det finns minst två av samma valör får de som har lagt den valör som det finns flest av spela
  \item Om det finns flera uppsättningar av samma antal kort med samma valör får de som har lagt den lägsta valören spela
  \item Om alla kort har olika värde går sticket till spelaren med det högsta kortet
\end{itemize}
Samma regler gäller när de inblandade spelarna lägger ett till kort.

\subsection{Slutet på första rundan}
Spelaren som drar det sista kortet (trumfkortet) ur högen lägger det åt sidan utan att titta på det tills alla har spelat ut sina kort. Då vänds det upp och färgen på kortet bestämmer vilken färg som är trumf. Det bestämmer även vilken färg som är anti-trumf; spader som trumf ger klöver som anti-trumf och tvärtom samt hjärter som trumf ger ruter som anti-trumf och tvärtom. Valören på trumfkortet bestämmer det antal kort varje spelare måste ha tagit för att undvika att få dela på skiten; valören multipliceras med antalet kortlekar. Ifall någon har mindre antal kort än vad trumfkortet angav slänger alla spelare in alla kort med valörerna 2-5. Dessa blandas och delas ut bland de som ska dela på skiten. Om trumfkortet är en joker räknas det som att det är ett ess och den som drar kortet får välja trumffärg utan att titta på sina kort.

\section{Andra rundan}
Den spelare som drog trumfkortet får börja.

\subsection{En spelares tur}
Spelaren har två alternativ; att plocka upp kort eller lägga kort.

\subsubsection{Plocka kort}
Spelaren plockar den lägsta ihophängande stegen (kan bestå av kort från en eller flera spelare).

\subsubsection{Lägga kort}
Spelaren lägger ett eller flera kort som bildar en stege (dubletter av samma kort får läggas). Följande regler gäller när det redan ligger kort på bordet (högre prio ju högre i listan):
\begin{itemize}
  \item En joker kan vara valfritt kort (ifall det inte är uppenbart vilket kort det är måste spelaren säga det)
  \item Ett par i trumf (med eventuell stigande stege) får läggas på vad som helst
  \item Trumf får inte läggas på anti-trumf
  \item Trumf får läggas på övriga färger och på lägre eller lika högt trumf
  \item En stege får börja på det som föregående slutade på
  \item En stege måste följa färg
\end{itemize}

\subsection{Vändning}
När det ligger lika många högar på bordet som det är spelare vänds korten bort och spelaren som vände får lägga valfria kort. När det vänds, byter man även håll som spelarna lägger kort (dvs. medurs blir moturs och tvärtom).

\subsection{Slutet på andra rundan}
När en spelare har lagt ut alla sina kort är denne gått ut. En spelare måste dock lägga ett kort efter denne har vänt. Har spelaren inte tillräckligt med kort är denne tvungen att plocka upp istället för att vända. Varje spelare måste dessutom säga ``Uno'' omgången före utgång, innan nästa spelare gjort sitt drag. När det endast finns en spelare kvar utnämns denne till skitgubbe.

\end{document}
